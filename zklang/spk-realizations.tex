% !TEX root =  main.tex


\label{spk-real}

In the following we will first explain how  protocols denoted
\[\textit{PK}
\{ (\alpha,\beta,\gamma): y = g^\alpha h^\beta \ \wedge \ \tilde{y} =
\tilde{g}^\alpha \tilde{h}^\gamma \ \wedge \ ( v < \alpha < u) \}
\]
can be constructed.
Unless otherwise stated, we assume a group $G = \langle g \rangle$ of prime order
$q$.

\subsection{Schnorr's Identification Scheme}
\label{Schnorr's Identification Scheme}
\index{Schnorr's identification scheme}
\index{proving knowledge of a discrete logarithm}

\begin{figure}[tb]
\begin{center}
\begin{tabular}{p{11em}cp{11em}}
\fbox{\ Prover\ } & & \hfill\fbox{\ Verifier\ }\\[1ex]
$(g,q,y,x)$ & & \hfill$(g,q,y)$\\
\hspace{1.5em}$\downarrow$ & & \hfill$\downarrow$\makebox[1.5em]{}\\[-1.6ex]\hline\\
%
$r_\alpha \in_R \Z{q}$\\[0.2ex]
$t := g^{r_\alpha}$& \LRarrow[15]{$t$}  & \\[-.5ex]
& & \hfill $c \in_R \{0,1\}^k$ \\[-.5ex]
& \LLarrow[15]{$c$} \\[-.5ex]
$s_\alpha := r_\alpha - c x \bmod{q}$ \\[-.5ex]
  & \LRarrow[15]{$s_\alpha$} \\[-.5ex]
&& \hfill $t \stackrel{?}{=} g^{s_\alpha} y^{c} $ \\[1ex]\hline\\[-3.8ex]
&& \hfill$\downarrow$\makebox[1.5em]{}\\
&&\hfill $(\mathtt{yes}/\mathtt{no})$ \\
\end{tabular}
\end{center}
\caption{
  The protocol denoted  $\textit{PK}\{(\alpha): y = g^\alpha \}$.
 The prover's input to the protocol is $(g,q,y,x)$ and the verifier's input 
is $(g,q,y)$. 
The prover has no output; the verifier's output is either $\mathtt{yes}$ or
$\mathtt{no}$, depending on whether or not he accepts the protocols, i.e., 
whether or not $t = g^{s_\alpha} y^{c} $ holds.
}

\label{fig:schnorr_id}

\end{figure}


The probably simplest is case is the protocol  
denoted $\textit{PK}\{(\alpha): y = g^\alpha \}$, where $y \in G$, and is depicted in Figure~\ref{fig:schnorr_id}.
This protocol is also known as Schnorr's identification protocol~\cite{schnor91}.
As the first step in the protocol,
the prover chooses a random integer $r_\alpha$, 
computes the \emph{protocol commitment}
\index{protocol commitment}
$t:= g^{r_\alpha}$ and sends it to the verifier.
The verifier replies with a random \emph{protocol challenge}
\index{protocol challenge}
 $c$. 
Next, the prover computes the \emph{protocol response} 
\index{protocol response} 
$s := r - cx \bmod{q}$ and sends it
to the verifier. 
Finally, the verifier accepts the protocol if the \emph{verification equation}
\index{verification equation}
$t = g^{s_\alpha} y ^c$ holds.

The protocol is a proof of knowledge of the discrete logarithm $\log_g y$ 
with cheating probability (knowledge error) of $2^{-k}$.
The protocol is also zero-knowledge against an honest verifier.

To achieve zero-knowledge against any verifier, one  needs to choose
$k$ logarithmic in the security parameter and repeat the protocols sufficiently 
many times to make the knowledge error small enough, losing some efficiency
by this repetition.
Reasonable parameters seem to be $k=10$ and repeating the protocol $8$ times to achieve an
overall cheating probability of $2^{-80}$.
Luckily, one can alternatively use one of the many known constructions to achieve zero-knowledge
that retain efficiency, e.g.,~\cite{damgar00}.
We note that this discussion holds for all the protocols we consider in this chapter.


From the protocol just discussed, one can derive the Schnorr signature scheme,
\index{Schnorr signature scheme}\index{SPK}
denoted  $\textit{SPK}\{(\alpha): y = g^\alpha \}(m)$, by
using the Fiat-Shamir heuristic \cite{fiasha86,poiste96a}, i.e., by
replacing the verifier by a call to a hash function $\mathcal{H}$ and 
thus computing the challenge as $c = \mathcal{H}(g\|y\|t\|m)$,
where $m \in \zo^*$ is the message that is signed.
The signature of $m$ consists of the pair $(s,c)$.
The verification equation of the signature scheme entails
computing $\hat{t} := g^s y^{c}$ and then verifying 
whether $c = \mathcal{H}(g\|y\|\hat{t}\|m)$ holds.
This signature scheme can be shown secure in the so-called random oracle 
model~\cite{belrog93}. 



\subsection{Proving Knowledge of a Representation}
\label{Proving Knowledge of a Representation}
\index{proving knowledge of a representation}

One generalization of Schnorr's identification scheme is to use, say, $\ell$ bases
$g_1, \ldots, g_\ell$  with $g_i \in G $.
That is, the protocol denoted $\textit{PK}\{(\alpha_1,\ldots, \alpha_\ell): y = 
\prod_{i =1}^{\ell}g_i^{\alpha_i}  \}$
is a proof of knowledge of the representation of $y \in G$ w.r.t.\ the bases $g_i$.
It is constructed as follows.
The inputs of the prover and the verifier consist of $y$, $g_1$, $\ldots$, $g_\ell$,  the order $q$
of the group, and the system parameter $k$.
The secret input of the prover consists of $x_i$'s  such that
$y = \prod_{i =1}^{\ell}g_i^{x_i}$.
Thus, each $x_i$ corresponds to an $\alpha_i$.
Now, to compute the 
first message of the protocol, the prover chooses $\ell$ random
integers $r_{\alpha_i} \in \Z{q}$,
computes $t := \prod_{i =1}^{\ell}g_i^{r_{\alpha_i}}$, and 
sends $t$ to the verifier.
The verifier replies with a  $c$ chosen as in the protocol above, i.e., randomly
from $\{0,1\}^k$.
Next, the prover computes
$s_{\alpha_i}  := r_{\alpha_i} - c x_{i} \bmod{q}$ and sends the resulting
$s_{\alpha_1},\ldots,s_{\alpha_\ell}$ to the verifier.
The verifier will accept the protocol if the equation 
$t = y^{c} \prod_{i =1}^{\ell}g_i^{s_{\alpha_i}}$
holds.
This protocol can be shown to be a proof of knowledge of values
$\alpha_i$ such that $ y =  \prod_{i =1}^{\ell}g_i^{\alpha_i}$
with knowledge error $2^{-k}$ and to be honest-verifier
zero-knowledge.

\subsection{Combining Different Proof Protocols}
\label{Combining Different Proof Protocols}

One can combine different instances of the protocol described so far.
The simplest combination is a protocol denoted  
\[
\textit{PK}\{(\alpha_1,\ldots, \alpha_\ell, \beta_1,\ldots, \beta_{\ell'} ): y = \prod_{i =1}^{\ell} g_i^{\alpha_i} 
\ \wedge \ z = \prod_{i =1}^{\ell'} h_i^{\beta_i}  \}
\]
%\end{multline*}
with $g_i,h_i,y,z \in G$.
It is obtained by running the two protocols 
\[\textit{PK}\{(\alpha_1,\ldots, \alpha_\ell,  ): y = \prod_{i =1}^{\ell} g_i^{\alpha_i}  \}
\quad \text{and} \quad
\textit{PK}\{(\beta_1,\ldots, \beta_{\ell'} ): 
 z = \prod_{i =1}^{\ell'} h_i^{\beta_i}  \}
\]
in parallel as sub-protocols as follows. 
First, the prover computes and  sends to the verifier
the commitment messages of both protocols at the same time.
Next, the verifier chooses and sends back a \emph{single} challenge message,
i.e., the verifier chooses the same challenge message for both protocols.
Finally, the verifier will accept the overall protocol only if the
verification equations of both (sub-)protocols hold.

In the same way one constructs a protocol that involves several terms (i.e., representations
of several values)
by just running the protocol for each term in parallel and by letting the 
challenge to be the same for each of these protocols.
So, for instance, the protocol
$\textit{PK}\{(\alpha, \beta, \gamma ): y = g^{\alpha} 
\ \wedge \ z =  h^\beta \ \wedge \ w = g^\gamma \}$
is obtained by running the three sub-protocols
$\textit{PK}\{(\alpha): y = g^{\alpha}  \}$,
$\textit{PK}\{( \beta ): z =  h^\beta  \}$, 
and 
$\textit{PK}\{(\gamma ):  w = g^\gamma \}$
in parallel in as just described.

\subsection{Proving Equality of Discrete Logarithms}
\label{Proving Equality of Discrete Logarithms}
\index{proving equality of discrete logarithms}



Another useful combination of the protocols discusses so far is one where,
 in addition to prove knowledge of discrete logarithms or representations, 
the prover can show that
some discrete logarithms are equal.
Such protocols are in principle also obtained from running the 
basic protocol for each term in parallel.
However, now not only are the challenges the same for each of these
protocols but also some of random choices of the prover as well as some 
of the response messages of the protocols needs to be the same.
In general, when we compose protocols for arbitrary terms,
the prover is to use the same random $r_\alpha$ in all the protocols
that involve $\alpha$ in their term.
Let us consider the protocol 
$\textit{PK}\{(\alpha,\beta): y = g^{\alpha} h^\beta \ \wedge \ z =  h^\alpha \}$
as an example.
The verifier's and the prover's common input to the protocol consists
of $y,z,g,h,q$ and the prover's secret input consists of
$x_\alpha$ and $x_\beta$ such that 
$y := g^{x_\alpha} h^{x_\beta}$ and  $z :=  h^{x_\alpha}$.
To compute the commitment message, the prover chooses  random
$r_\alpha$ and $r_\beta$ from $\Z{q}$ and computes
$t_y := g^{r_\alpha} h^{r_\beta}$ and
$t_z := h^{r_\alpha}$.
Upon receiving the commitment messages $t_y$ and $t_z$, the verifier 
replies with a single  random challenge message $c \in_R \{0,1\}^k$. 
Next, the prover computes the response messages as
$s_\alpha := r_\alpha + c x_\alpha \bmod{q}$
and
$s_\beta := r_\beta + c x_\beta \bmod{q}$.
Having received $s_\alpha$ and $s_\beta$, the 
verifier will accept the protocol if 
the two verification equations
$t_y = y^c g^{s_\alpha} h^{s_\beta} $
and 
$t_z =  z^c h^{s_\alpha}$
hold.

Let us explain why the verifier should be convinced 
that $\log_h z$ equals the first element in the representation 
of $y$ w.r.t.\  $g$ and $h$.
Using standard rewinding techniques, one can obtain from 
a successful prover commitment and response messages 
for different challenge messages but the same commitment messages,
i.e., two tuples $(t_y,t_z,c,s_\alpha,s_\beta)$
and $(t'_y,t'_z,c',s'_\alpha,s'_\beta)$ with
$(t_y,t_z) = (t'_y,t'_z)$ and $c\neq c'$.
From the verification equations one can thus conclude that
\[
 y^{c-c'} =  g^{s'_\alpha-s_\alpha} h^{s'_\beta- s_\beta } 
\quad \text{and }\quad
 z^{c-c'} =  h^{s'_\alpha-s_\alpha}\enspace.
\]
Now we can set $\hat{x}_\alpha := (s'_\alpha-s_\alpha) (c-c')^{-1} \bmod{q}$
and $\hat{x}_\beta := (s'_\beta-s_\beta) (c-c')^{-1} \bmod{q}$
and thus we have
\[
 y =  g^{\hat{x}_\alpha} h^{\hat{x}_\beta } 
\quad \text{and }\quad
 z =  h^{\hat{x}_\alpha}\enspace,
\]
i.e., we see that indeed $\log_h z$ equals the first element in the representation 
of $y$ w.r.t.\  $g$ and $h$.

From what we have now seen, we are able to construct protocols
that fall into the class denoted
\begin{multline*}
\mathit{PK}\{(\alpha_1, \ldots \alpha_{\ell_\alpha}):
y_1 = \prod_{i \in I_1 } g_i^{\alpha_{f_1(i)}}
\ \wedge\ \
y_2 = \prod_{i \in I_2 } g_i^{\alpha_{f_2(i)}}
\ \wedge\ \ 
\cdots
\ \wedge\ \
y_{\ell_y} = \prod_{i \in I_{\ell_y} } g_i^{\alpha_{f_{\ell_y}(i)}}
\}\enspace,
\end{multline*}
where 
\begin{itemize}
\item $\ell_\alpha$, $\ell_g$, and  $\ell_y$ are parameters denoting 
the number of secrets $\alpha_i$, of bases $g_i$, and of elements $y_i$, respectively,
\item the $y_i$'s and $g_i$'s can be arbitrary elements of $G$ and do not necessarily be
distinct or can be a product of other (given) elements, e.g., $y_3 = y_5 g_2^3 / y_2$, 
\item the sets $I_j$ define which bases are used for the $j$-th
term and the functions $f_j$ define which secret $\alpha_k$ is used
for a particular base in the term.
\end{itemize}
The protocol is depicted in Figure~\ref{fig:combinded-protocol}.
We note that the protocol has the property that from a successful prover
one can extract values $\alpha_1, \ldots, \alpha_{\ell_\alpha}$
such that the equations
\[
y_1 = \prod_{i \in I_1 } g_i^{\alpha_{f_1(i)}}, \quad
y_2 = \prod_{i \in I_2 } g_i^{\alpha_{f_2(i)}}, \quad
\cdots, \quad 
y_{\ell_y} = \prod_{i \in I_{\ell_y} } g_i^{\alpha_{f_{\ell_y}(i)}}
\]
all hold.
That is, it is an honest-verifier zero-knowledge proof of knowledge of the values $\alpha_1, \ldots, \alpha_{\ell_\alpha}$
with knowledge error $2^{-k}$.


Let us finally consider some example instances of this general protocol.
The protocol denoted
$\mathit{PK}\{(\alpha,\beta): y_1 = g^{\alpha} \wedge y_2 = g^{\beta} 
\wedge y_3 = y_1^{\beta}\}$
proves that 
$\log_g y_3$ is the product of $\log_g y_1$ and $\log_g y_2$,
the protocol denoted 
$\mathit{PK}\{(\alpha): y_1 = g^{\alpha} \wedge y_2 = y_1^{\alpha}\}$
proves that $\log_g y_2 = (\log_g y_1)^2$,
and 
the protocol denoted
$\mathit{PK}\{(\alpha,\beta): g = y^{\alpha} h^{\beta}\}$
proves that the first element of the representation of $y$
that the prover knows w.r.t.\ $g$ and $h$ is non-zero, provided that the prover is not privy
to $log_g h$
(in case the prover knows $log_g h$, he is able to compute $q$ different 
representations of $y$ w.r.t.\ $g$ and $h$, otherwise he can only know one).



\begin{figure}[tb]
\begin{center}
\begin{tabular}{p{11em}cp{11em}}
\fbox{\ Prover\ } & & \hfill\fbox{\ Verifier\ }\\[1ex]
$(g_1,\ldots,g_{\ell_g},
y_1,\ldots,y_{\ell_y},
$ & & \hfill
$(g_1,\ldots,g_{\ell_g},
y_1,\ldots,y_{\ell_y},
$
\\
$I_1,\ldots,I_{\ell_y},
f_1,\ldots,f_{\ell_y},
$ & & \hfill$
I_1,\ldots,I_{\ell_y},
f_1,\ldots,f_{\ell_y},
$
\\
$
x_1,\ldots,x_{\ell_\alpha},
q)$ & & \hfill$
q)$
\\
\hspace{1.5em}$\downarrow$ & & \hfill$\downarrow$\makebox[1.5em]{}\\[-1.6ex]\hline\\
%
$r_{\alpha_1},\ldots,r_{\alpha_{\ell_\alpha}} \in_R \Z{q}$\\[0.2ex]
For $j = 1, \ldots, \ell_y$ do: \\
\mbox{}\hfill 
$t_{y_j} := \prod_{i \in I_j } g_i^{r_{\alpha_{f_j(i)}}} $
& \LRarrow[15]{$t_{y_1},\ldots,t_{y_{\ell_y}}$}  & \\[-.5ex]
& & \hfill $c \in_R \{0,1\}^k$ \\[-.5ex]
& \LLarrow[15]{$c$} \\[-.5ex]
$s_{\alpha_i} := r_{\alpha_i} - c x_i \bmod{q}$ \\[-.5ex]
  & \LRarrow[15]{$s_{\alpha_1},\ldots,s_{\alpha_{\ell_\alpha}}$} \\[-.5ex]
&& For $j = 1, \ldots, \ell_y$ check: \\
&& \hfill 
$t_{y_j} \stackrel{?}{=} y_j ^{c} \prod_{i \in I_j } g_i^{s_{\alpha_{f_j(i)}}}  $ \\[1ex]\hline\\[-3.8ex]
&& \hfill$\downarrow$\makebox[1.5em]{}\\
&&\hfill $(\mathtt{yes}/\mathtt{no})$ \\
\end{tabular}
\end{center}
\caption{
The protocol
denoted $\mathit{PK}\{(\alpha_1, \ldots \alpha_{\ell_\alpha}):
y_1 = \prod_{i \in I_1 } g_i^{\alpha_{f_1(i)}}
\ \wedge\ \
y_2 = \prod_{i \in I_2 } g_i^{\alpha_{f_2(i)}}
\ \wedge\ \
\cdots
\ \wedge\ \
y_{\ell_y} = \prod_{i \in I_{\ell_y} } g_i^{\alpha_{f_{\ell_y}(i)}}
\}$.
}

\label{fig:combinded-protocol}

\end{figure}



%\work{\subsection{Linear Equations }
%\label{Linear Equations }
%
%We note that we can also add constants, i.e., the protocol 
%$\mathit{PK}\{(\alpha): y_1 = g_1^{\alpha} \wedge y_2 = g_2^{3\alpha}\}$
%is obtained by interpreting the second base as $g_2^3$, i.e.,  
%$\mathit{PK}\{(\alpha): y_1 = g_1^{\alpha} \wedge y_2 = (g_2^3)^{\alpha}\}$.
%}


\subsection{The Schnorr Protocol Modulo a Composite.}  
\label{The Schnorr Protocol Modulo a Composite.}  
\label{sec:proto}
\index{unkown order}

So far we have considered proof protocols for a group of prime order where the order is
known to both the prover and the verifier.
However, often one would like to use these kind of protocols in groups
where the order is not known to all parties as is for instance the case
in RSA-groups.
\index{RSA group}
RSA groups are subgroups of $\Zs{n}$, where $n$ is the product of two primes.
Thus, if one does not know the factorization of $n$, one does in general
not know the order of a subgroup generated by a random element of $\Zs{n}$.

In particular, let $n$ be the product of two safe primes, i.e., primes $p$ and $q$ such that
$p'=(p-1)/2$ and $q'=(q-1)/2$ are also primes.\index{safe prime product}
Let $g$ be a generator of the quadratic residues modulo $n$ (so, $g$ will have order
$p'q'$).\index{quadratic residue}
In this case, knowing the order $p'q'$ of $g$ is equivalent to knowing the factorization of $n$.
Let $y=g^x$ with $x \in \{0,1\}^\ell$ for some $\ell$.
Now consider the protocol denoted $\textit{PK}\{(\alpha): y = g^\alpha \pmod{n}\}$, where 
at least the prover is assumed not to be  privy to the factorization of $n$.
The prover and verifier's common inputs are $(y,g,n,\ell_n)$ and the prover's additional input is
$x = \log_g y$, where $\ell_n = \lceil \log_2 n \rceil$ (i.e., the length of $n$ in bits). 
We may assume that $x \in [0,n]$.
The protocol is as follows.
\begin{enumerate}
\item The prover chooses uniformly at random $r_\alpha  \in_R \{0,1\}^{\ell_n+\ell_c + \ell_\varnothing}$,
computes $t_y := g^r \bmod{n}$, and sends $t$ to the verifier.
\item The verifier chooses a random $c \in_R \{0,1\}^{\ell_c}$ and sends that to the 
prover.
\item The prover replies with $s_\alpha := r_\alpha - xc$.
\item The verifier checks whether $t_y \stackrel{?}{\equiv} y^c g^{s_\alpha} \pmod{n}$ holds.
\end{enumerate}

The difference to the protocol in the case the order of the group $\langle g \rangle$ 
is known is
that here, as she does not know the order of the group,
the prover can no longer choose $r_\alpha$ randomly from the integers modulo this 
order and can no longer reduce $s_\alpha$ modulo this order.
So, the prover needs to choose $r_\alpha$ some how differently
such that nevertheless $s_\alpha$ and $t_y$ 
do not reveal information about $x$, i.e., such that
the protocol remains zero-knowledge.
Thus if $x  \in [0,n]$ and the prover chooses $r_\alpha \in_R \{0,1\}^{\ell_n+\ell_c + \ell_\varnothing}$,
then, for any  $c \in \{0,1\}^{\ell_c}$ and sufficiently large $\ell_\varnothing$ (e.g., 80), 
the value $s_\alpha := r_\alpha - xc$ is 
distributed statistically close to the uniform distribution 
over $\{0,1\}^{\ell_n+\ell_c + \ell_\varnothing}$.
Also, the value $t_y := g^r \bmod{n}$ is distributed statistically close to uniform 
over $\langle g \rangle$.
Therefore, provided that $y \in  \langle g \rangle$, the protocol is  statistical honest-verifier zero-knowledge 
for sufficiently large $\ell_\varnothing$ (e.g., $\ell_\varnothing = 80 $).



Now, this protocol is only a proof of knowledge of $\log_g y$ if $\ell_c$ equals
$1$ and if repeated sufficiently many, say $k$, times.
Let us investigate this.
Assume we are given a prover that can successfully run the protocol for given
$y$, $g$, and $n$.
By standard rewinding techniques, one can extract two triples
$(t,c,s)$ and $(t',c',s')$ from the prover such that 
$t=t'$, $c \neq c'$, and $t \equiv y^c g^s  \equiv y^{c'} g^{s'} \pmod{n}$
holds. 
W.l.o.g.\ we may assume that $c'>c$.
From the last equation we can derive that
$y^{c'-c} \equiv g^{s-s'} \pmod{n}$.

If $\ell_c =1$, we must have $c' = 1$ and $c=0$ and
hence $y \equiv g^{s-s'} \pmod{n}$, i.e., $s-s'$ is a discrete logarithm
of $y$ to the base $g$ and hence the protocol is indeed a proof of knowledge.

If  $\ell_c > 1 $, we are stuck with the equation $y^{c'-c} \equiv g^{s-s'} \pmod{n}$, i.e., we seem to need to compute a $(c'-c)$-th root of  $g^{s-s'}$ modulo $n$ which 
is assumed to be hard without knowledge of $g$'s order.
Unfortunately,  Theorem~\ref{thm:strong-rsa} provides a way out.
\index{Strong RSA assumption}
That is, under the strong RSA assumption, we will have that 
$(c'-c) | (s-s')$. 
Let $u$ be such that $(c'-c) u = (s-s')$.
Then $y \equiv b g^u \pmod{n}$ for some 
$b$ such that $b^{c'-c} \equiv 1 \pmod{n}$.
Assuming that $2^\ell_c < \min(p',q')$, it must be 
that $b = \pm 1$ or  $\gcd(b \pm 1,n)$ splits $n$ (which again would counter
the strong RSA assumption).
Of course, if both $y$ and $g$ are quadratic residues, then
$y \equiv g^u \pmod{n}$. 

The reader might now think that the protocol is therefore a proof of knowledge under
the strong RSA assumption for $\ell_c > 1$.
Unfortunately this is not the case.
Let us expand.
The definition of a proof of knowledge
(cf.~Chapter~\ref{chapter:7}), 
requires that the inputs $n$, $g$, and $y$ be fixed, i.e., that
the knowledge extractor works for any prover that is successful for a 
given input $n$, $g$, and $y$.
However, the above argument only works for $n$ chosen at random.
For instance, it does not work if the prover knew the factorization of 
$n$ as he then could compute $c$, $c'$, $s$, and $s'$ such that
$(c'-c) \nmid (s-s')$ (for instance by adding a multiple of the order of 
$g$ to $s$).
Now, if $n$ is fixed, there always exists a prover who has the factorization 
encoded into herself and thus she could successfully run the protocol but the
knowledge extractor cannot extract a witness.
In order words, the protocol only has the proof of knowledge property 
for random $n$ which contradicts the requirement of a proof of knowledge that 
the witness can be extracted for any given $n$.
Nevertheless, the protocol is still useful as a building block, i.e., 
one only need to take into account the probability spaces of $n$, $g$, and $y$.
That is, one has to consider the overall system of which the protocol is part and cannot just
consider the protocol by itself as one could if it was a true proof of knowledge.
Despite of all of this, we denote this protocol also as 
$\textit{PK}\{(\alpha): y \equiv \pm g^\alpha \pmod{n}\}$ or
$\textit{PK}\{(\alpha): y \equiv g^\alpha \pmod{n}\}$, depending on whether the verifier
is assured that $y$ is a quadratic residue or not.

\subsection{Proving that a Secret Lies in a Given Interval}
\label{Proving that a Secret Lies in a Given Interval}
\index{proving that a secret lies in a given interval}
\index{interval proof}

One property of  the protocol just described that is handy in many cases is 
the fact that the prover cannot reduce the response messages modulo the 
order of the group is to argue about the bit-length of the secret known to the prover.
Let $x \in \pm \{0,1\}^{\ell_x}$ for some $\ell_x$ and let $y := g^x \bmod{n}$, with 
$g$ and $n$ as before.
Now consider the following modification of the protocol 
(the inputs to the prover and the verifier
remain unchanged except that both parties are further given $\ell_x$).
\begin{enumerate}
\item The prover chooses uniform at random $r_\alpha  \in_R \{0,1\}^{\ell_x+\ell_c + \ell_\varnothing}$,
computes $t_y := g^r \bmod{n}$, and sends $t$ to the verifier.
\item The verifier chooses a random $c \in_R \{0,1\}^{\ell_c}$ and sends that to the 
prover.
\item The prover replies with $s_\alpha := r_\alpha - xc$.
\item The verifier checks whether 
\[
t_y \stackrel{?}{\equiv} y^c g^{s_\alpha} \pmod{n}
\quad\text{and}\quad
s_\alpha \stackrel{?}{\in} \pm \{0,1\}^{\ell_x+\ell_c + \ell_\varnothing+1}
\]
hold.
\end{enumerate}
The modification is that the prover chooses $r_\alpha$ from a different interval
and that the verifier also checks that $s_\alpha$ takes at most 
${\ell_x+\ell_c + \ell_\varnothing+1}$ bits.

The analysis of this protocols is of course almost identical to the original one, except
that we are now considering the bit-lengths of $s_\alpha$ and $r_\alpha$.
First, it is not difficult to see that the protocol remains statistical 
honest-verifier zero-knowledge.
Above we have argued that under the strong RSA assumption, one can extract
a value $u$ from a successful prover such that 
$y \equiv \pm g^u \pmod{n}$,
where with $u = (s-s')/(c'-c)$. 
Now because  $(c'-c)$ divides $(s-s')$ and as 
$(s-s') \in \pm\{0,1\}^{\ell_x+\ell_c + \ell_\varnothing+2}$,
we must have
$u \in \pm\{0,1\}^{\ell_x+\ell_c + \ell_\varnothing+2}$.
In other words, the discrete logarithm known to the prover has at most 
$\ell_x+\ell_c + \ell_\varnothing+2$ bits (neglecting its sign). 
Note that this length is independent of the length of the modulus $n$.
Also note that in fact the length of the prover's secret must be shorter, 
(only about $\ell_x$ bits) for the prover to be able to run
the protocol successfully with high probability; so the protocol is not an argument of the
exact length of the secret.
However, in many cases this is good enough. 
We denote this modified protocol as 
$\textit{PK}\{(\alpha): y \equiv g^\alpha \pmod{n}  \quad \wedge \quad 
\alpha \in  \pm\zo^{\ell_x+\ell_c + \ell_\varnothing+2}\}$.

The protocol can be also be used to prove that the secret known to the prover 
lies in any interval, say, in $[a,b]$.
To this end note that $[a,b] = [\frac{a+b}{2} -  \frac{b-a}{2} , \frac{a+b}{2} + \frac{b-2}{2} ]$.
Thus the protocol denoted 
\[\textit{PK}\{(\alpha): \frac{y}{g^{(a+b)/2}} \equiv g^\alpha \pmod{n} \quad \wedge \quad \alpha \in 
[-  \frac{b-a}{2} , \frac{b-a}{2} ] 
\}
\]
is an argument that the prover knows a value $x$ such 
that $x \in [a,b]$ and $y \equiv g^x \pmod{n}$.
As before, the actual value $x$ given as input to the prover 
must lie in a smaller interval, namely
in the interval 
$[\frac{a+b}{2} -  \frac{b-a}{2 \cdot 2^{\ell_c + \ell_\varnothing +2}} , 
\frac{a+b}{2} + \frac{a+b}{2 \cdot 2^{\ell_c + \ell_\varnothing +2}}]$.


It is straightforward to extent and generalize the protocols just discussed 
in the same way as the
protocols we considered for groups of 
known order.
Moreover, the protocols over groups of known order and those over groups of
unknown order can be combined.
Let us consider a simple combination as an example; 
the constructions of general combinations
is left as an exercise to the reader.

Assume a group $G = \langle g \rangle$ of order a prime $q$,
and let $n$ be an RSA modulus as above,  $h_1$ be an element
from $\Zs{n}$, $h_2$ be an element from $\langle h_1 \rangle$, and
$y = g^x$ with $x \in \zo^{\ell_x}$ for some integer $\ell_x < (\log_2 q) - 1 - ( \ell_c 
+ \ell_\varnothing + 2) $. 
Finally, assume that the prover is not privy to $n$'s factorization.
Now consider the following protocol.
The common input to the prover and the verifier consists of $q$, $g$, $y$, $n$, $h_1$,  $h_2$, and $a$
and the secret input to the prover consists of $x$. 
\begin{enumerate}
\item First, the prover chooses a random $r \in_R [0,n 2^{\ell_\varnothing}]$, computes
$z := h_1^x h_2^r \bmod{n}$, and sends $z$ to the verifier.
\item Next, the prover and the verifier run the protocol denoted:
\begin{multline*}
\textit{PK}\{(\alpha,\beta ): y = g^\alpha 
\ \ \wedge \ \
 z \equiv h_1^\alpha h_2^\beta \!\!\!\!\pmod{n} \ \ \wedge \ \
\alpha \in  \pm\zo^{\ell_x+\ell_c + \ell_\varnothing+2}\},
\end{multline*}
i.e., they perform the following 
steps.
\begin{enumerate}
\item The prover chooses a random $r_\alpha \in [-a2^{\ell_c + \ell_\varnothing}, a2^{\ell_c + \ell_\varnothing}] $
and a random $r_\beta \in \zo^{\ell_n+\ell_c + 2 \ell_\varnothing}] $, computes
$t_y := g^{r_\alpha}$ and $t_z := h_1^{r_\alpha} h_2^{r_\beta} \bmod{n}$ and 
sends $t_y$ and $t_z$ to the verifier.
\item The verifier chooses a random $c \in_R \{0,1\}^{\ell_c}$ and sends that to the 
prover.
\item The prover replies with $s_\alpha := r_\alpha - xc$ and 
$s_\beta := r_\beta - r c$.
\item The verifier checks whether 
\begin{xalignat*}{2}
t_y & \stackrel{?}{=}  y^c g^{s_\alpha} \quad, &
t_z &\stackrel{?}{\equiv}  z^c h_1^{s_\alpha} h_2^{s_\beta} \pmod{n}
\quad, \ \text{and}\quad \\
s_\alpha &\stackrel{?}{\in}  \pm \{0,1\}^{\ell_x+\ell_c + \ell_\varnothing+1}
\end{xalignat*}
hold.
\end{enumerate}
\end{enumerate}

Let us analyze this protocol.
From our considerations above we know that 
one can extract from a successful prover
values $(t_y,t_z, c, s_\alpha, s_\beta)$
and  $(t'_y,t'_z, c', s'_\alpha, s'_\beta)$
with $(t_y,t_z) = (t'_y,t'_z)$ and $c \neq c'$.
From the verification equations we have that 
\begin{xalignat*}{2}
z^{c-c'} &\equiv h_1^{s'_\alpha - s_\alpha} h_2^{s'_\beta - s_\beta} \pmod{n}\enspace, \quad &
y^{c-c'} &= g^{s'_\alpha - s_\alpha}\enspace, \quad \text{and}\\
(s'_\alpha - s_\alpha) &\in \pm \{0,1\}^{\ell_x+\ell_c + \ell_\varnothing+2}\enspace.
\end{xalignat*}
From the first  of these equations we can conclude that under 
the Strong RSA assumption $(c-c')$ divides $(s'_\alpha - s_\alpha)$, i.e.,
there exists some $u $ such that $(s'_\alpha - s_\alpha) = u (c-c') $.
  Thus, we can rewrite the second equation as 
$y^{c-c'} = g^{u (c-c')}$.
Now, as $y \in \langle g \rangle$ (which we can test as $g$ has prime order $q$),
we must have $y = g^u$.
From the third of the equations we can further
 derive that $u \in  \pm \{0,1\}^{\ell_x+\ell_c + \ell_\varnothing+2}$.
Thus the verifier is assured that the prover knows $\log_g y$ which lies in
 $\pm \{0,1\}^{\ell_x+\ell_c + \ell_\varnothing+2}$, i.e., the protocol 
is a proof that a discrete logarithm in a group of \emph{known order} lies in some interval.
Of course this makes  sense only if the group's order 
is larger than $2^{\ell_x+\ell_c + \ell_\varnothing+3}$.

As the basic protocol $\textit{PK}\{(\alpha): y \equiv \pm g^\alpha \pmod{n}\}$ in 
group of unknown order, the protocol just described in not a proof of knowledge either
as its analysis depends on the strong RSA assumption and thus is correct only if the prover can 
execute the protocol for a random $n$.
As the goal of the protocol was to prove that $\log_g y$ lies in some interval,
we do not need to fix $n$ but could have the verifier generate $n$ and send it to the
prover before running the protocol.
The protocol augmented like this would indeed be a true proof of knowledge.
For the protocol to be zero-knowledge, the prover needs to be ensured that $h_2 \in \langle h_1 \rangle$ as
otherwise $z$ could leak information about $x$.
Unfortunately, the only way known to prove that $h_2 \in \langle h_1 \rangle$ holds is
not very efficient, i.e., is to have the verifier run the protocol
$\textit{PK}\{(\alpha): h_2 \equiv \pm h_1^\alpha \pmod{n}\}$ many times with binary challenges 
(cf.\ above).
However, in some cases this proof can be delegated to a set-up phase
and thus needs to be done only once and for all,
or sometimes $n$, $h_1$, and $h_2$ can be provided by a trusted third party.


The protocol just discussed can be extended to three (or more) different groups, i.e., 
groups $G_1 = \langle g_1 \rangle$  and $G_2 = \langle g_2 \rangle$  of known order
$q_1$ and $q_2$ together with a RSA sub-group of unknown order as to show that two 
discrete logarithms in $G_1$ and $G_2$ are the same.
I.e., let $y_1 = g_1^x$ and $y_2 = g_2^x$ with $x \in \pm\zo^{\ell_x}$ and
 $\ell_x < (\log_2 \min\{q_1,q_2\}) - 1 - ( \ell_c 
+ \ell_\varnothing + 2) $, and $z = h_1^x h_2^r \pmod{n}$ for some random $r$.
Then, the protocol
\begin{multline*}
\textit{PK}\{(\alpha,\beta ): y_1 = g_1^\alpha  
\quad \wedge \quad 
y_2 = g_2^\alpha  
\quad \wedge \quad  \
 z \equiv h_1^\alpha h_2^\beta \pmod{n}  \quad \wedge \quad 
\alpha \in  \pm\zo^{\ell_x+\ell_c + \ell_\varnothing+2}\}\enspace,
\end{multline*}
will convince a verifier that $\log_{g_1} y_1 = \log_{g_2} y_1 $ 
where we define $\log_g y$ to be the integer $x$ for which $y=g^x$
and that closed to $0$, i.e.,  $\log_g y$ lies in $[-q/2,q/2]$.
Of course, the protocol can also be generalized to representations, i.e., to a protocol
showing that an element of a representation lies in a certain interval.

%- add attributes
%- add anonymity revocation

