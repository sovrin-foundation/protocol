\documentclass[a4paper]{article}
\usepackage[utf8]{inputenc}
\usepackage{geometry}
\usepackage{amssymb,amsthm,amsmath,amsfonts,longtable,comment,array,ifpdf,hyperref,url}
\usepackage{graphicx}
\usepackage{authblk}

\input{preamble}


\title{Zero Knowledge Language}
\author[1]{Michael Lodder}
\author[1]{Brent Zundel}
\author[1]{Jason Law}
\author[1]{Lovesh Harchandani}
\author[2]{Jan Camenisch}
\author[2]{Manu Drijvers}

\affil[1]{Evernym}
\affil[2]{IBM Zurich}

\date{10th January 2018, version 0.1}

\begin{document}

\maketitle

\section{Introduction}
Currently there exists many different methods for describing zero knowledge proofs in cryptographic mathematical notation. Zero Knowledge Language or ZKLang attempts to create a formal language to describe basic programmer mappings of zero knowledge proofs to complex cryptographic algorithms.~\footnote{\url{https://docs.google.com/presentation/d/1Zt1oRXASoKo03mnDpG7kNhjMDv6v4GhfIPNO6_aTEYM/edit?usp=sharing}}

\section{Background}

\subsection{Mathmatical symbols}
TODO: Evernym

\subsection{Set Theory}
TODO: Evernym

\subsection{Sigma protocols}
TODO: Evernym

\section{Terminology}
TODO: Both Evernym and IBM
%\input{terminology}

% !TEX root =  main.tex

We could do all of this with X.509 credentials, but then have no privacy features. 
We here concentrate on how to do this with the privacy features.


We assume that the system parameters describe a groups $\G_1$, $\G_2$, $\G_T$, of prime order $q$, with efficiently computable bilinear map $e$.
We further assume here that all attributes $m_i$ are elements of $Z_q$, and consider the encoding of other typed attribute values in different sections.

\subsection{Generalized Schnorr Protocols Notation}


In the following we will use various protocols to prove knowledge of and relations 
among discrete logarithms.
To describe these protocols, we use  notation introduced by Camenisch and
Stadler~\cite{camsta97b} for various proofs of knowledge of discrete
logarithms and proofs of the validity of statements about discrete logarithms.
For instance,
\[\textit{SPK}
\{ (\alpha,\beta,\gamma): y = g^\alpha h^\beta \ \wedge \ \tilde{y} =
\tilde{g}^\alpha \tilde{h}^\gamma \ \wedge \ ( v < \alpha < u) \}
\]
denotes a ``\textit{zero-knowledge \emph{P}roof of \emph{K}nowledge of
  integers $\alpha$, $\beta$, and $\gamma$ such that $y = g^\alpha h^\beta$
  and $\tilde{y} = \tilde{g}^\alpha \tilde{h}^\gamma$ holds, with $ v <
  \alpha < u$},'' where $y,g,h,\tilde{y},\tilde{g}$, and $\tilde{h}$ are
elements of some groups $G = \langle g\rangle = \langle h\rangle$ and
$\tilde{G} = \langle \tilde{g}\rangle = \langle \tilde{h} \rangle$.  
The
convention is that Greek letters denote the quantities the knowledge of which is
being proved, while all other parameters are known to the verifier. 
More precisely, the property of ``proof of knowledge'' means that there exists a
\emph{knowledge extraction} 
\index{knowledge extraction}
algorithm who can 
extract the Greek quantities from a successful prover if given rewinding and
reset access to the prover (cf.~Appendix~\ref{spk-real}).
Thus, using this
notation, a proof protocol can be described by just pointing out its aim while
hiding the protocols realization details.



\subsection{Realization of Credential $\operatorname{Credential}$ - CL Signatures}
A credential will take the form of a signature created by an issuer. 
As we want to prove knowledge of credentials, we need ``signatures with efficient protocols'', also called CL signatures~\cite{camlys02b}. 
Examples are the RSA-based CL signature~\cite{camlys02b}, the pairing-based CL signature~\cite{camlys04},
and the BBS+ signature scheme~\cite{bobosh04,ASMC13}. 
We recall the BBS+ signature scheme:
\begin{description}
\item[Key Generation] Take $(h_0, \ldots, h_L) \gets \G_1^{L+1}$, $x \gets Z_q^*$, $w \gets g_2^x$, and set $sk = x$ and $pk = (w, h_0, \ldots, h_L)$.
\item[Signature] On input message $(m_1, \ldots, m_L) \in Z_q^L$ and secret key $x$, 
pick $e, s \gets Z_q$ and compute $A \gets (g_1 h_0^s \prod_{i = 1}^{L} h_i^{m_{i}})^{\frac{1}{e+x}}$.
Output signature $\sigma \gets (A, e, s)$.
\item[Verification] On input a public key $(w, h_0, \ldots, h_L) \in \G_2 \times \G_1^{L+1}$, 
message $(m_1, \ldots, m_L) \in Z_q^L$, and purported signature $(A, e, s) \in \G_1 \times Z_q^2$, 
check $e(A, w g_2^e) = e(g_1 h_0^s \prod_{i = 1}^{L} h_i^{m_{i}}, g_2)$.
\end{description}

We can use the following zero-knowledge proof to prove knowledge of a BBS+ signature, while selectively disclosing the attributes~\cite{CDL16eprint}:
The prover has signature $\sigma \gets (A, e, s)$ with $A = (g_1 h_0^s \prod_{i = 1}^{L} h_i^{m_{i}})^{\frac{1}{e+x}}$.
He can prove knowledge of a BBS+ signature while selectively disclosing messages $m_i$ with $i \in D$.
Randomize the credential by taking $r_1 \gets Z_q^*$, set $A' \gets A^{r_1}$, and set $r_3 \gets \frac{1}{r_1}$.
Set $\bar{A} \gets A'^{-e} \cdot b^{r_1} (= A'^{x})$.
Choose $r_2 \gets Z_p$, set $d \gets (g_1 h_0^s \prod_{i = 1}^{L} h_i^{m_{i}})^{r_1} \cdot h_0^{-r_2}$, and set $s' \gets s - r_2 \cdot r_3$.
The prover now proves 
\begin{multline*}
\pi \in \SPK\{(\{m_i\}_{i \notin D}, e, r_2, r_3, s') : \\
\bar{A}/d = A'^{-e} \cdot h_0^{r_2} ~\land~ g_1 \prod_{i \in D} {h_{i}}^{m_i}= d^{r_3}h_0^{-s'} \prod_{i \notin D} {h_{i}}^{-m_i}\}.
\end{multline*}
The resulting proof consists of $(A', \bar{A}, d, \pi)$.
To verify a proof, the verifier checks $A' \neq 1_{\G_1}$, $e(A', w) = e(\bar{A}, g_2)$, and verifies $\pi$.

\subsection{Pseudonyms}
Pseudonyms can be constructed from Pedersen commitments~\cite{peders91b} to the user secret key.
A user can create arbitrarily many pseudonyms, all of which are unlinkable.

Let $g_1$ and $h_1$ be generators of $\G_1$. 
Pedersen commitments are constructed as follows.
\begin{description}
\item[Commit] To commit to a value $m_1 \in Z_q$, take $r \gets Z_q$ and output $c \gets g_1^{m_1}h_1^r$.
\item[ComVf] To verify that $c$ commits to $m_1$ with opening $r$, check $c \stackrel{?}{=} g_1^{m_1}h_1^r$.
\end{description}

One can efficiently prove that a pseudonym $\nym$ is correctly constructed by proving 
\[\SPK\{(m_1, r) : \nym = g_1^{m_1} h_1^r \}.\]

For certain applications, we might want to restrict a user to create only a single pseudonym for one context.
Domain specific pseudonyms offer this functionality. 
Let $\mathit{bsn}$ be a string that describes the context, a scope exclusive pseudonym can be constructed by $\nym \gets H(\mathit{bsn})^{m_1}$. 
A user can prove this is correctly computed:
\[\SPK\{(m_1) : \nym = H(\mathit{bsn})^{m_1} \} .\]


\subsection{Range proofs}


\subsection{Verifiable Encryption}
Verifiable encryption allows a prover to prove that a certain value is correctly encrypted to a certain public key.
Here, we consider verifiable encryption of group elements for which the prover knows $b^m$, for some base $b \in \G_1$.

Recall the Elgamal encryption scheme: 
to encrypt $m \in \G_1$ to public key $\mathit{epk}$, take $r \gets Z_q$, and set $c_1 \gets g_1^r$, $c_2 \gets \mathit{epk}^r \cdot m$.

To verifiably encrypt a value $b^m$ to public key $\mathit{epk}$, the prover creates an Elgamal ciphertext $(c_1, c_2)$ encrypting $b^m$ to $\mathit{epk}$,
and proves
\[\SPK\{(m_1, r) : c_1 = g_1^r \land c_2 = \mathit{epk}^r b^{m_1} \}.\]

\subsection{Orchestration}



\subsection{Mapping Verifiable Claims to ZKLang}
TODO: IBM

\subsection{Mapping the different types to integers}
TODO: IBM

\subsection{Age proof}
TODO: IBM

\subsection{Membership proof}
TODO: IBM

\subsection{CL signatures}
TODO: IBM

\subsection{Pseudonyms,Commitments}
TODO: IBM

\subsection{Range proofs}
TODO: IBM

\subsection{Verifiable Encryption}
TODO: IBM

\subsection{Orchestration}
TODO: IBM

\section{Statements}
TODO: Both Evernym and IBM

%\input{statements}

\subsection{Realization of ZKLang Components}
TODO: IBM

\end{document}