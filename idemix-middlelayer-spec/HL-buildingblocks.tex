% !TEX root =  PrivateCredSpecification.tex



\section{ZKLang}

\newcommand{\NIZK}{\operatorname{NIZK}}

If credentials are key-bound, they are required to be bound to the same (secret) key. 

At this level, all message $m_i$ are integers.

\begin{multline}
\NIZK\lbrace(m_i)_{i\in h}[m]_{i\not\in h}: \operatorname{Credential}( \emph{ipk}, m_1, m_2, m_3) \rbrace\shoveleft\
\end{multline}


\begin{multline}
\NIZK\lbrace(): \operatorname{Nym}( \emph{nym}) \rbrace\shoveleft\
\end{multline}


\begin{multline}
\NIZK\lbrace(): \operatorname{SNym}( \emph{nym}, \mathit{scope}) \rbrace\shoveleft\
\end{multline}


\begin{multline}
\NIZK\lbrace(m): \operatorname{Enc}( \emph{epk}, m, \mathit{ctxt}) \rbrace\shoveleft\
\end{multline}

\begin{multline}
\NIZK\lbrace(m): \operatorname{Larger}(m, c) \rbrace\shoveleft\
\end{multline}

\begin{multline}
\NIZK\lbrace(m): \operatorname{Smaller}(m, c) \rbrace\shoveleft\
\end{multline}


Example composition: here 

Explanations of stuff



\section{Mapping Verifiable Claims to ZKLang}




\section{Realization of ZKLang Components}

$m_i$ from $Z_q$, so everything in prime order group

Nyms

CL sigs

Vereng

Orchestration
