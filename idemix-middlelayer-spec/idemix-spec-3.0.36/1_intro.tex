%!TEX root =  IdmxSpecification.tex

\section{Introduction}
\label{sec:introduction}

The amount of our daily transactions that we perform electronically is increasing drastically.
Many of us use the Internet on a daily basis for 
purposes ranging from accessing information to electronic commerce and e-banking
to interactions with government bodies. 
Securing these transactions requires the use of strong authentication. 
Electronic authentication tokens and mechanisms
that provide authentication become common, not only for the use with the Internet. 
Indeed, electronic identity cards, authentication by mobile phone,
and RFID tokens are spreading fast. 


These authentication mechanisms unfortunately
have the shortcoming that they label users with a unique identifier.
This is a risk to users' privacy because   
transactions by the same user can be \emph{linked} together.
This lack of privacy is typically not a problem for e-government
applications. 
However, a government-issued strong root of trust is very
attractive for re-use by the commercial sector as it alleviates them from having to issue credentials themselves. 
In this application area, unique identification is often inappropriate, attribute-based
authentication desired and privacy important to make the services
sustainable. 
Therefore, these services and their authentication mechanisms should be
built in a way that minimizes the disclosed personal information.
A position paper issued in February 2009 by ENISA%
\footnote{ENISA: European Network and Information Security Agency \url{http://www.enisa.europa.eu}} 
on
``Privacy Features of European eID Card Specifications'' underlines the need for
``privacy-respecting use of unique identifiers'' in emerging European eID cards, and explicitly refers to the
emerging anonymous credentials technologies (``privacy-enhanced PKI tokens'' in
their terminology), as having significant potential in this arena.
\includeThis{
Indeed, over the past decades, the research community has come up with a
large number of privacy-enhancing technologies that can be employed to 
this end.}


Anonymous credentials~\cite{brands95b,brands95c,camlys01a,chaum85,damgar88},
allow an identity provider to issue a credential (also called certificate) to a user. 
This credential contains attributes of a user such as her address or date of birth but
also the user's rights or roles. 
Using the credential, the user can later prove to a third party that she possesses 
a credential containing a given
attribute or role without revealing any other information stored in the
credential. 
For instance, the user could employ a government-issued anonymous ID
credential to prove that she is of age, i.e., that she possesses a credential
containing a date of birth attribute, which lies further than 21 years in the past. 
Thus, anonymous credentials promise to be an important technology in protecting
users' privacy in an electronic environment.
In particular, even if she uses the same credential repeatedly, the different usages cannot be
linked to each other.


There is a large body of research on anonymous credential systems and a number of
different methods or algorithms are described in the literature. 
In addition to the basic functionality of an anonymous credential system, i.e., 
to issue a credential and then later to selectively reveal attributes contained in 
credentials, there are extensions proposed in the literature to
meet the requirements of a real world deployment. 
These extensions include the  revocation of
credentials~\cite{brdede07,cakoso09,camlys04,nfhf09}, revocation of 
anonymity~\cite{camlys01a}, encoding attributes pertaining to a finite set of values
efficiently~\cite{camgro08}, or to verifiably encrypt attributes under some third
party's encryption key~\cite{camdam00,camsho03}.
Further, it has been demonstrated, that anonymous credentials can be used in practice 
today - even on smart cards~\cite{bcgs09}.
These important features were described independently, with different setup
assumptions and trust models.


Based on these papers we have implemented a unified system called the the {\em
Identity Mixer Anonymous Credential System}, which takes the form of a
cryptographic library.  
Here we presents the full specification with all cryptographic details.  
It is  similar to a cryptographic
library that offers, for example, implementations of the RSA or DSA signature
schemes: it offers all the functionalities required to establish a
pseudonym, issue a credential containing different attributes to a pseudonym,
and different ways of proving possession of a credential. 
The ways to prove possession of a credential offered are the disclosure of a 
selected subset of the
attributes contained in the credential, to prove that an (integer) attribute
lies in a given range, to prove that an attribute is verifiably encrypted under
some third party's public key, or that a cryptographic commitment contains a
specific attribute.  
The source code of the new implementation of the Identity Mixer 
library can be downloaded from \footnote{\url{https://abc4trust.eu/idemix/}}.


\includeThis{
The literature provides a number of cryptographic building blocks that (can be employed to) extend 
this basic functionality, many of which are needed to meet the practical requirements of a modern public key infrastructure.
These include:
\begin{description}
% \item[\normalfont\emph{Compact encoding}\hspace{-.5ex}] of attributes, \ie,  in case 
% attributes take on only a few values, they can be encoded 
% very efficiently representing each value as a unique prime~\cite{camgro08}.
% Additionally, this representation allows for proving that an attribute is a 
% member of a set;
\item[\normalfont\emph{Property proofs}\hspace{-.5ex}] about attributes allow a 
credential owner to prove properties about her attributes such as 
that one attribute is larger than another one (even if they are contained in
different credentials), which allows to prove, \eg, that her age 
lies in a certain range~\cite{cachsh08}, or that
an attribute is a member of a given set~\cite{camgro08}. 
% \item[\normalfont\emph{Commitments to attributes}\hspace{-.5ex}] that can be used later 
% on for application specific purposes.
% For example, commitments allow a user to attain a credential on a value without the
% issuer learning this value.
% an issuer can ensure that a newly issued credential 
% contains (some of) the same attributes as a credential that a user already possesses 
% without the issuer of the new credential learning the attribute value;
% \item[\normalfont\emph{Domain pseudonyms}\hspace{-.5ex}] ensure that a single 
% user can only have one pseudonym per domain. 
% Note that pseudonyms from different domains cannot be linked.
\item[\normalfont\emph{Usage limitation}\hspace{-.5ex}] such as ensuring that a user can 
show a 
credential only a limited number of times (e.g., for e-cash)~\cite{caholy05} or a number 
of times within some context~\cite{chklm06,caholy06}, \eg, twice per hour or once per 
election.
Furthermore, using domain specific pseudonyms allow to implement usage restriction 
as it makes a user linkable within a given domain.
\item[\normalfont\emph{Revocation of credentials}\hspace{-.5ex}] in case of leakage of 
secret values of the credential can be implemented using dynamic 
accumulators~\cite{cakoso09,camlys04} or
a form of credential revocation lists~\cite{bonsha04,brdede07,nfhf09}.
\item[\normalfont\emph{Revocation of anonymity}\hspace{-.5ex}] in case of abuse of (the 
rights granted by) a credential~\cite{camlys01a}.
\item[\normalfont\emph{Verifiable encryption}\hspace{-.5ex}] of attributes under some 
third party's public key~\cite{camsho02}, which 
can be seen as a generalization of anonymity revocation (the user's identity could be an 
attribute encrypted for the party in charge of anonymity revocation) and allows to 
control the dispersal of attributes.
\end{description}
These mechanisms can be combined in various ways and thereby they allow us
to build various privacy-enhancing applications such as anonymous e-cash, petition
systems, pseudonymous reputations systems, or anonymous and oblivious access control 
systems. 
The protocols of the resulting full-fledged anonymous credential system become complex
and it is challenging to design an architecture and application interfaces for
its implementation.
}

