%!TEX root =  IdmxSpecification.tex

\section{Overview of an Anonymous Credential System}
\label{sec:credSysOverview}

A basic anonymous credential system involves the roles of \emph{issuers}, \emph{recipients},
\emph{provers} and \emph{verifiers} (or \emph{relying parties}).
Parties acting in those roles execute the issuing protocol, where a credential for the
recipient is created by the issuer, or the proving protocol, where the owner creates a proof
on behalf of the verifier. 
An entity (\eg, user, company, government) can assume any role during each protocol run. 
For instance, a company can act as verifier and run the proof protocol with a user before
assuming the role of the issuer and running the issuance protocol (possibly with the same
user). 
Finally, an extended credential system requires the role of trusted third parties who 
performs tasks such as anonymity revocation, credential revocation, or decryption
of (verifiably) encrypted attributes.
Usually organizations or governments assume the roles the issuer, verifier and trusted party,
and natural persons the ones of recipient and prover.

Note, all parties in an anonymous credential system agree on general system parameters that
define the bit length of all relevant parameters as well as the groups that will be used. 
In practice, these parameters can be distributed together with the code and they must be 
authenticated.

To participate a user needs to choose her \emph{secret key} based on the group
parameters of the system.
This secret allows her to derive pseudonyms, which she can use similar to a session
identifier, \ie, it allows the communication partner to link the actions of the user.
However, the user can create new pseudonyms at her discretion and all pseudonyms are
unlinkable unless the user proves that they are based on the same secret key.
Certain scenarios require one user only having one pseudonym with an organization,
where we call such pseudonym a scope exclusive pseudonym.
In addition to being used for pseudonym generation, the master secret will be encoded into
every credential.
This constitutes a sharing prevention mechanism as sharing one credential implies sharing all
credentials of a user.

The setup procedure for issuers and trusted parties consists of generating public key pairs,
create a specification of the services they offer and publish the specification as well as the
public key.
As an example, an issuer runs the issuer key generation and publishes the structure(s) of the
credential(s) it is willing to issue together with its public key.

Let us now elaborate on the issuing and the proving protocol.
The credential \emph{issuance protocol} is carried out between an issuer and a recipient with
the result of the recipient having a credential.
The credential consists of a set of attribute values as well as cryptographic information
that allows the owner of the credential (i.e., the recipient) to create a \emph{proof of
possession} (also called `proof of ownership' or `proof'). 
When encoding the values into a credential, the issuer and recipient agree on which values the
issuer learns and which will remain unknown to it, \ie, they agree on a credential
structure. 
The issuer may require to learn partial information about the values that he does not learn.
If this is required, the parties run a proving protocol in which the issuer acts as verifier and the
recipient as prover before interacting in the issuance protocol.
We call such a transaction \emph{advanced issuance}.


The \emph{proving protocol} requires a prover and a verifier to interact, \ie, the owner of
one or several credentials acts as prover in the communication with a verifier. 
Firstly, the verifier defines what statement will be proved about which
attribute value.
Secondly, the prover compiles a cryptographic proof that complies with the statements 
negotiated before.
Thirdly, the verifier checks if the given proof is compiled correctly. 
The first step is a very elaborate process that is outside of the scope of this paper.
To indicate the complexity remember that a proof can range from merely proving possession of
a credential issued by some issuer to proving detailed statements about the individual
attributes. 
Our specification uses the ABC4Trust languages that express the results possibly negotiated by the 
entities and focuses on the second and third step mentioned before. 
The difficulties here lie in the fact that a proof may be linked to a pseudonym of the user's
choice or it may release a verifiable encryption of some attribute value under a third
party's public key.
In addition, we need to be able to express statements about attributes that will be 
proved.
Finally, the protocols for proving possession of credentials and issuing
credentials may be combined.
In particular, before issuing a new credential, the issuer may require the recipient 
to release certified attribute values, \ie, prove that she holds a credential issued by
another party.

